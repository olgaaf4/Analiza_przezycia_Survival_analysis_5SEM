% Options for packages loaded elsewhere
\PassOptionsToPackage{unicode}{hyperref}
\PassOptionsToPackage{hyphens}{url}
%
\documentclass[
  11pt,
]{article}
\usepackage{amsmath,amssymb}
\usepackage{iftex}
\ifPDFTeX
  \usepackage[T1]{fontenc}
  \usepackage[utf8]{inputenc}
  \usepackage{textcomp} % provide euro and other symbols
\else % if luatex or xetex
  \usepackage{unicode-math} % this also loads fontspec
  \defaultfontfeatures{Scale=MatchLowercase}
  \defaultfontfeatures[\rmfamily]{Ligatures=TeX,Scale=1}
\fi
\usepackage{lmodern}
\ifPDFTeX\else
  % xetex/luatex font selection
\fi
% Use upquote if available, for straight quotes in verbatim environments
\IfFileExists{upquote.sty}{\usepackage{upquote}}{}
\IfFileExists{microtype.sty}{% use microtype if available
  \usepackage[]{microtype}
  \UseMicrotypeSet[protrusion]{basicmath} % disable protrusion for tt fonts
}{}
\makeatletter
\@ifundefined{KOMAClassName}{% if non-KOMA class
  \IfFileExists{parskip.sty}{%
    \usepackage{parskip}
  }{% else
    \setlength{\parindent}{0pt}
    \setlength{\parskip}{6pt plus 2pt minus 1pt}}
}{% if KOMA class
  \KOMAoptions{parskip=half}}
\makeatother
\usepackage{xcolor}
\usepackage[margin=1in]{geometry}
\usepackage{color}
\usepackage{fancyvrb}
\newcommand{\VerbBar}{|}
\newcommand{\VERB}{\Verb[commandchars=\\\{\}]}
\DefineVerbatimEnvironment{Highlighting}{Verbatim}{commandchars=\\\{\}}
% Add ',fontsize=\small' for more characters per line
\usepackage{framed}
\definecolor{shadecolor}{RGB}{248,248,248}
\newenvironment{Shaded}{\begin{snugshade}}{\end{snugshade}}
\newcommand{\AlertTok}[1]{\textcolor[rgb]{0.94,0.16,0.16}{#1}}
\newcommand{\AnnotationTok}[1]{\textcolor[rgb]{0.56,0.35,0.01}{\textbf{\textit{#1}}}}
\newcommand{\AttributeTok}[1]{\textcolor[rgb]{0.13,0.29,0.53}{#1}}
\newcommand{\BaseNTok}[1]{\textcolor[rgb]{0.00,0.00,0.81}{#1}}
\newcommand{\BuiltInTok}[1]{#1}
\newcommand{\CharTok}[1]{\textcolor[rgb]{0.31,0.60,0.02}{#1}}
\newcommand{\CommentTok}[1]{\textcolor[rgb]{0.56,0.35,0.01}{\textit{#1}}}
\newcommand{\CommentVarTok}[1]{\textcolor[rgb]{0.56,0.35,0.01}{\textbf{\textit{#1}}}}
\newcommand{\ConstantTok}[1]{\textcolor[rgb]{0.56,0.35,0.01}{#1}}
\newcommand{\ControlFlowTok}[1]{\textcolor[rgb]{0.13,0.29,0.53}{\textbf{#1}}}
\newcommand{\DataTypeTok}[1]{\textcolor[rgb]{0.13,0.29,0.53}{#1}}
\newcommand{\DecValTok}[1]{\textcolor[rgb]{0.00,0.00,0.81}{#1}}
\newcommand{\DocumentationTok}[1]{\textcolor[rgb]{0.56,0.35,0.01}{\textbf{\textit{#1}}}}
\newcommand{\ErrorTok}[1]{\textcolor[rgb]{0.64,0.00,0.00}{\textbf{#1}}}
\newcommand{\ExtensionTok}[1]{#1}
\newcommand{\FloatTok}[1]{\textcolor[rgb]{0.00,0.00,0.81}{#1}}
\newcommand{\FunctionTok}[1]{\textcolor[rgb]{0.13,0.29,0.53}{\textbf{#1}}}
\newcommand{\ImportTok}[1]{#1}
\newcommand{\InformationTok}[1]{\textcolor[rgb]{0.56,0.35,0.01}{\textbf{\textit{#1}}}}
\newcommand{\KeywordTok}[1]{\textcolor[rgb]{0.13,0.29,0.53}{\textbf{#1}}}
\newcommand{\NormalTok}[1]{#1}
\newcommand{\OperatorTok}[1]{\textcolor[rgb]{0.81,0.36,0.00}{\textbf{#1}}}
\newcommand{\OtherTok}[1]{\textcolor[rgb]{0.56,0.35,0.01}{#1}}
\newcommand{\PreprocessorTok}[1]{\textcolor[rgb]{0.56,0.35,0.01}{\textit{#1}}}
\newcommand{\RegionMarkerTok}[1]{#1}
\newcommand{\SpecialCharTok}[1]{\textcolor[rgb]{0.81,0.36,0.00}{\textbf{#1}}}
\newcommand{\SpecialStringTok}[1]{\textcolor[rgb]{0.31,0.60,0.02}{#1}}
\newcommand{\StringTok}[1]{\textcolor[rgb]{0.31,0.60,0.02}{#1}}
\newcommand{\VariableTok}[1]{\textcolor[rgb]{0.00,0.00,0.00}{#1}}
\newcommand{\VerbatimStringTok}[1]{\textcolor[rgb]{0.31,0.60,0.02}{#1}}
\newcommand{\WarningTok}[1]{\textcolor[rgb]{0.56,0.35,0.01}{\textbf{\textit{#1}}}}
\usepackage{graphicx}
\makeatletter
\def\maxwidth{\ifdim\Gin@nat@width>\linewidth\linewidth\else\Gin@nat@width\fi}
\def\maxheight{\ifdim\Gin@nat@height>\textheight\textheight\else\Gin@nat@height\fi}
\makeatother
% Scale images if necessary, so that they will not overflow the page
% margins by default, and it is still possible to overwrite the defaults
% using explicit options in \includegraphics[width, height, ...]{}
\setkeys{Gin}{width=\maxwidth,height=\maxheight,keepaspectratio}
% Set default figure placement to htbp
\makeatletter
\def\fps@figure{htbp}
\makeatother
\setlength{\emergencystretch}{3em} % prevent overfull lines
\providecommand{\tightlist}{%
  \setlength{\itemsep}{0pt}\setlength{\parskip}{0pt}}
\setcounter{secnumdepth}{5}
\usepackage[OT4]{polski}
\usepackage[utf8]{inputenc}
\usepackage{graphicx}
\usepackage{float}
\ifLuaTeX
  \usepackage{selnolig}  % disable illegal ligatures
\fi
\usepackage{bookmark}
\IfFileExists{xurl.sty}{\usepackage{xurl}}{} % add URL line breaks if available
\urlstyle{same}
\hypersetup{
  pdftitle={Lista 5},
  pdfauthor={Olga Foriasz, Tomasz Warzecha},
  hidelinks,
  pdfcreator={LaTeX via pandoc}}

\title{Lista 5}
\usepackage{etoolbox}
\makeatletter
\providecommand{\subtitle}[1]{% add subtitle to \maketitle
  \apptocmd{\@title}{\par {\large #1 \par}}{}{}
}
\makeatother
\subtitle{Analiza przeżycia}
\author{Olga Foriasz, Tomasz Warzecha}
\date{2025-10-08}

\begin{document}
\maketitle

{
\setcounter{tocdepth}{2}
\tableofcontents
}
\section{Lista 5}\label{lista-5}

\subsection{Zadanie 1 - Analiza graficzna estymatorów funkcji
przeżycia}\label{zadanie-1---analiza-graficzna-estymatoruxf3w-funkcji-przeux17cycia}

Celem zadania było graficzne przedstawienie i porównanie estymatorów
funkcji przeżycia dla czasu do remisji choroby w grupach pacjentów
leczonych lekiem A oraz lekiem B. Dane pobrano z zadania 3 z listy 2.
Wykorzystano dwie metody nieparametryczne:

\begin{enumerate}
\def\labelenumi{\alph{enumi})}
\tightlist
\item
  Estymator Kaplana-Meiera (KM) b) Estymator Fleminga-Harringtona (FH)
\end{enumerate}

\includegraphics{Analiza_przeżycia_5_files/figure-latex/unnamed-chunk-3-1.pdf}
Na powyższym wykresie widzimy estymator Kaplana-Meinera dla grup A i B.

Teraz wyznaczmy sobie estymator Fleminga-Harringtona:
\includegraphics{Analiza_przeżycia_5_files/figure-latex/unnamed-chunk-4-1.pdf}
Na powyższym wykresie widzimy estymator Fleminga-Harringtona dla grup A
i B. Zarówno wykres estymatora Kaplana-Meiera, jak i
Fleminga-Harringtona, są wizualnie niemal identyczne i prowadzą do tych
samych wniosków.

Na podstawie wykresów estymatorów Kaplana--Meiera można zauważyć, że
krzywe przeżycia dla grup A i B mają zbliżony przebieg. Początkowo
wartości funkcji przeżycia są prawie identyczne, natomiast w środkowym
zakresie czasów (około 0.3-0.7) krzywa dla grupy B przebiega nieco
niżej, co sugeruje, że pacjenci leczeni lekiem B częściej osiągali
remisję wcześniej niż pacjenci z grupy A. Ostatecznie jednak obie krzywe
schodzą do podobnego poziomu, co wskazuje, że końcowy odsetek remisji w
obu grupach jest zbliżony. Zatem na podstawie otrzymanych wykresów można
przypuszczać, że leki A i B wykazują podobne działanie, przy czym lek B
może powodować nieco szybsze wystąpienie remisji we wcześniejszym
okresie obserwacji.

\subsection{Zadanie 2 - tworzenie wykresu estymatora Kaplana-Meiera z
``ogonem''}\label{zadanie-2---tworzenie-wykresu-estymatora-kaplana-meiera-z-ogonem}

Na początku stworzymy sobie funkcję do tworzenia wykresu estymatora
Kaplana-Meiera z ``ogonem'' estymowanym zgodnie z propozycją Browna,
Hollandera i Kowara. W tym celu na początku wyznaczymy sobie estymator
Kaplana-Meiera za pomocą funkcji \(survfit()\) dla \(t \in (0, t^+]\),
następnie dla \(t \in (t^+, 2t^+)\) wyznaczymy sobie ``ogon'' naszego
estymatora za pomocą wzoru: \[
\hat{S}(t) = \exp \left[ \frac{\ln \hat{S}(t^+)}{t^+} t \right],
\]

\begin{Shaded}
\begin{Highlighting}[]
\NormalTok{est }\OtherTok{\textless{}{-}} \ControlFlowTok{function}\NormalTok{(time,km,t0)\{}
\NormalTok{  y }\OtherTok{\textless{}{-}} \FunctionTok{numeric}\NormalTok{(}\FunctionTok{length}\NormalTok{(time))}
\NormalTok{  i }\OtherTok{\textless{}{-}} \DecValTok{1}
  \ControlFlowTok{for}\NormalTok{ (t }\ControlFlowTok{in}\NormalTok{ time)\{}
\NormalTok{    y[i] }\OtherTok{\textless{}{-}} \FunctionTok{exp}\NormalTok{(}\FunctionTok{log}\NormalTok{(}\FunctionTok{min}\NormalTok{(km))}\SpecialCharTok{*}\NormalTok{t}\SpecialCharTok{/}\NormalTok{t0)}
\NormalTok{    i }\OtherTok{=}\NormalTok{ i}\SpecialCharTok{+}\DecValTok{1}
\NormalTok{  \}}
  \FunctionTok{return}\NormalTok{(y)}
\NormalTok{\}}

\NormalTok{wykres }\OtherTok{\textless{}{-}} \ControlFlowTok{function}\NormalTok{(dane,t0,nazwa\_wykresu)\{}
\NormalTok{  km\_fit }\OtherTok{\textless{}{-}} \FunctionTok{survfit}\NormalTok{(}\FunctionTok{Surv}\NormalTok{(czas,delta)}\SpecialCharTok{\textasciitilde{}}\DecValTok{1}\NormalTok{, dane)}
\NormalTok{  ogonx }\OtherTok{\textless{}{-}} \FunctionTok{seq}\NormalTok{(}\AttributeTok{from =}\NormalTok{ (t0), }\AttributeTok{to =} \DecValTok{2}\SpecialCharTok{*}\NormalTok{t0, }\AttributeTok{by =} \FloatTok{0.1}\NormalTok{)}
\NormalTok{  ogony }\OtherTok{\textless{}{-}} \FunctionTok{est}\NormalTok{(ogonx,km\_fit}\SpecialCharTok{$}\NormalTok{surv,t0)}
  \FunctionTok{plot}\NormalTok{(km\_fit, }\AttributeTok{col =} \StringTok{"darkblue"}\NormalTok{,}\AttributeTok{conf.int =} \ConstantTok{FALSE}\NormalTok{,}
       \AttributeTok{xlab =} \StringTok{"Czas"}\NormalTok{, }\AttributeTok{ylab =} \StringTok{"Prawdopodobieństwo przeżycia"}\NormalTok{, }
       \AttributeTok{main =}\NormalTok{ nazwa\_wykresu,}
       \AttributeTok{xlim =} \FunctionTok{c}\NormalTok{(}\DecValTok{0}\NormalTok{, t0}\SpecialCharTok{*}\DecValTok{2}\NormalTok{))}
  \FunctionTok{lines}\NormalTok{(ogony, }\AttributeTok{col =} \StringTok{"lightblue"}\NormalTok{, }\AttributeTok{lwd =} \DecValTok{2}\NormalTok{)}
  \FunctionTok{legend}\NormalTok{(}\StringTok{"topright"}\NormalTok{,}
         \AttributeTok{legend =} \FunctionTok{c}\NormalTok{(}\StringTok{"Estymator KM"}\NormalTok{, }\StringTok{"\textquotesingle{}ogon\textquotesingle{}"}\NormalTok{),}
         \AttributeTok{col =} \FunctionTok{c}\NormalTok{(}\StringTok{"darkblue"}\NormalTok{, }\StringTok{"lightblue"}\NormalTok{),}
         \AttributeTok{lwd =} \DecValTok{2}\NormalTok{)}
\NormalTok{\}}
\end{Highlighting}
\end{Shaded}

Teraz mając funkcję tworzącą wykresy, narysujemy sobie wykres naszego
estymatora dla danych z zad.3 z listy 2.

\begin{Shaded}
\begin{Highlighting}[]
\FunctionTok{par}\NormalTok{(}\AttributeTok{mfrow =} \FunctionTok{c}\NormalTok{(}\DecValTok{1}\NormalTok{,}\DecValTok{2}\NormalTok{))}
\FunctionTok{wykres}\NormalTok{(dane\_A,}\DecValTok{1}\NormalTok{,}\StringTok{"Estymator KM z ogonem dla gr. A"}\NormalTok{)}

\FunctionTok{wykres}\NormalTok{(dane\_B,}\DecValTok{1}\NormalTok{,}\StringTok{"Estymator KM z ogonem dla gr. B"}\NormalTok{)}
\end{Highlighting}
\end{Shaded}

\includegraphics{Analiza_przeżycia_5_files/figure-latex/unnamed-chunk-6-1.pdf}
Na powyższych wykresach możemy zobaczyć wartości estymatora K-M z ogonem
dla grupy A i B. Jak widać powyższe wykresy znacznie się nie różnią,
ogon estymatora Kaplana--Meiera pokazuje dalszy spadek funkcji przeżycia
po ostatnich zaobserwowanych zdarzeniach, co oznacza, że
prawdopodobieństwo przeżycia (braku remisji) nadal maleje w miarę upływu
czasu, mimo braku nowych obserwacji zdarzeń.

\subsection{Zadanie 3 - oszacowanie wartości funkcji
przeżycia}\label{zadanie-3---oszacowanie-wartoux15bci-funkcji-przeux17cycia}

\begin{Shaded}
\begin{Highlighting}[]
\NormalTok{typ\_I }\OtherTok{\textless{}{-}} \ControlFlowTok{function}\NormalTok{(t0, alfa, lambda, n)\{}
\NormalTok{  p }\OtherTok{\textless{}{-}} \FunctionTok{runif}\NormalTok{(n)}
\NormalTok{  X}\OtherTok{\textless{}{-}}\FunctionTok{numeric}\NormalTok{(n)}
\NormalTok{  delta }\OtherTok{\textless{}{-}}\FunctionTok{numeric}\NormalTok{(n)}
  \ControlFlowTok{for}\NormalTok{(i }\ControlFlowTok{in} \DecValTok{1}\SpecialCharTok{:}\NormalTok{n)\{}
\NormalTok{  X[i]}\OtherTok{=}\NormalTok{(}\SpecialCharTok{{-}}\NormalTok{(}\DecValTok{1}\SpecialCharTok{/}\NormalTok{lambda))}\SpecialCharTok{*}\FunctionTok{log}\NormalTok{(}\DecValTok{1}\SpecialCharTok{{-}}\NormalTok{(p[i])}\SpecialCharTok{\^{}}\NormalTok{(}\DecValTok{1}\SpecialCharTok{/}\NormalTok{alfa))}
  \ControlFlowTok{if}\NormalTok{(X[i]}\SpecialCharTok{\textgreater{}}\NormalTok{t0)\{}
\NormalTok{    X[i] }\OtherTok{\textless{}{-}}\NormalTok{ t0}
\NormalTok{    delta[i] }\OtherTok{\textless{}{-}} \DecValTok{0}
\NormalTok{  \}}\ControlFlowTok{else}\NormalTok{\{delta[i]}\OtherTok{\textless{}{-}}\DecValTok{1}\NormalTok{\}\}}
  \FunctionTok{return}\NormalTok{(}\FunctionTok{data.frame}\NormalTok{(}\AttributeTok{X =}\NormalTok{ X, }\AttributeTok{delta =}\NormalTok{ delta))}
\NormalTok{\}}

\NormalTok{est }\OtherTok{\textless{}{-}} \ControlFlowTok{function}\NormalTok{(time,km,t0)\{}
\NormalTok{  y }\OtherTok{\textless{}{-}} \FunctionTok{numeric}\NormalTok{(}\FunctionTok{length}\NormalTok{(time))}
\NormalTok{  i }\OtherTok{\textless{}{-}} \DecValTok{1}
  \ControlFlowTok{for}\NormalTok{ (t }\ControlFlowTok{in}\NormalTok{ time)\{}
\NormalTok{    y[i] }\OtherTok{\textless{}{-}} \FunctionTok{exp}\NormalTok{(}\FunctionTok{log}\NormalTok{(}\FunctionTok{min}\NormalTok{(km))}\SpecialCharTok{*}\NormalTok{t}\SpecialCharTok{/}\NormalTok{t0)}
\NormalTok{    i }\OtherTok{=}\NormalTok{ i}\SpecialCharTok{+}\DecValTok{1}
\NormalTok{  \}}
  \FunctionTok{return}\NormalTok{(y)}
\NormalTok{\}}

\NormalTok{est\_KM\_ogon }\OtherTok{\textless{}{-}} \ControlFlowTok{function}\NormalTok{(dane,t0)\{}
\NormalTok{  km\_fit }\OtherTok{\textless{}{-}} \FunctionTok{survfit}\NormalTok{(}\FunctionTok{Surv}\NormalTok{(X,delta)}\SpecialCharTok{\textasciitilde{}}\DecValTok{1}\NormalTok{, dane)}
\NormalTok{  ogonx }\OtherTok{\textless{}{-}} \FunctionTok{seq}\NormalTok{(}\AttributeTok{from =}\NormalTok{ (t0}\FloatTok{+0.1}\NormalTok{), }\AttributeTok{to =} \DecValTok{2}\SpecialCharTok{*}\NormalTok{t0, }\AttributeTok{by =} \FloatTok{0.1}\NormalTok{)}
\NormalTok{  ogony }\OtherTok{\textless{}{-}} \FunctionTok{est}\NormalTok{(ogonx,km\_fit}\SpecialCharTok{$}\NormalTok{surv,t0)}
\NormalTok{  fy }\OtherTok{\textless{}{-}} \FunctionTok{c}\NormalTok{(km\_fit}\SpecialCharTok{$}\NormalTok{surv, ogony)}
\NormalTok{  fx }\OtherTok{\textless{}{-}} \FunctionTok{c}\NormalTok{(km\_fit}\SpecialCharTok{$}\NormalTok{time, ogonx)}
\NormalTok{  funkcja }\OtherTok{\textless{}{-}} \FunctionTok{data.frame}\NormalTok{(}\AttributeTok{x =}\NormalTok{ fx, }\AttributeTok{y =}\NormalTok{ fy)}
\NormalTok{  xt0 }\OtherTok{\textless{}{-}}\NormalTok{ funkcja}\SpecialCharTok{$}\NormalTok{y[funkcja}\SpecialCharTok{$}\NormalTok{x }\SpecialCharTok{==}\NormalTok{ t0]}
\NormalTok{  x2t0 }\OtherTok{\textless{}{-}}\NormalTok{ funkcja}\SpecialCharTok{$}\NormalTok{y[funkcja}\SpecialCharTok{$}\NormalTok{x }\SpecialCharTok{==} \DecValTok{2}\SpecialCharTok{*}\NormalTok{t0]}
  \FunctionTok{return}\NormalTok{(}\FunctionTok{c}\NormalTok{(xt0,x2t0))}
  
\NormalTok{\}}
\FunctionTok{par}\NormalTok{(}\AttributeTok{mfrow =} \FunctionTok{c}\NormalTok{(}\DecValTok{1}\NormalTok{,}\DecValTok{2}\NormalTok{))}


\NormalTok{histogramy }\OtherTok{\textless{}{-}} \ControlFlowTok{function}\NormalTok{(alfa,lambda,t0,m,n\_list)\{}
  \ControlFlowTok{for}\NormalTok{(n }\ControlFlowTok{in}\NormalTok{ n\_list)\{}
\NormalTok{    st0 }\OtherTok{\textless{}{-}} \FunctionTok{numeric}\NormalTok{(m)}
\NormalTok{    s2t0 }\OtherTok{\textless{}{-}} \FunctionTok{numeric}\NormalTok{(m)}
    \ControlFlowTok{for}\NormalTok{(i }\ControlFlowTok{in}\NormalTok{ (}\DecValTok{1}\SpecialCharTok{:}\NormalTok{m))\{}
\NormalTok{      dane }\OtherTok{\textless{}{-}} \FunctionTok{typ\_I}\NormalTok{(t0,alfa,lambda,n)}
\NormalTok{      s }\OtherTok{\textless{}{-}} \FunctionTok{est\_KM\_ogon}\NormalTok{(dane,t0)}
\NormalTok{      st0[i]}\OtherTok{\textless{}{-}}\NormalTok{ s[}\DecValTok{1}\NormalTok{]}
\NormalTok{      s2t0[i]}\OtherTok{\textless{}{-}}\NormalTok{ s[}\DecValTok{2}\NormalTok{]}
\NormalTok{    \}}
    \FunctionTok{hist}\NormalTok{(st0, }\AttributeTok{col =} \StringTok{"lightblue"}\NormalTok{, }\AttributeTok{main =} \FunctionTok{paste}\NormalTok{(}\StringTok{"n ="}\NormalTok{, n, }\StringTok{" | Estymator S(t0)"}\NormalTok{), }\AttributeTok{xlab =} \StringTok{"Wartość S(t0)"}\NormalTok{)}
    \FunctionTok{hist}\NormalTok{(s2t0, }\AttributeTok{col =} \StringTok{"aquamarine"}\NormalTok{,}\AttributeTok{main =} \FunctionTok{paste}\NormalTok{(}\StringTok{"n ="}\NormalTok{, n, }\StringTok{" | Estymator S(2 * t0)"}\NormalTok{), }\AttributeTok{xlab =} \StringTok{"Wartość S(2 * t0)"}\NormalTok{)}
\NormalTok{  \}}
\NormalTok{\}}

\NormalTok{n\_list }\OtherTok{\textless{}{-}} \FunctionTok{c}\NormalTok{(}\DecValTok{30}\NormalTok{,}\DecValTok{50}\NormalTok{,}\DecValTok{100}\NormalTok{)}
\NormalTok{alfa }\OtherTok{\textless{}{-}} \DecValTok{4}
\NormalTok{lambda }\OtherTok{\textless{}{-}} \DecValTok{1}
\NormalTok{t0 }\OtherTok{\textless{}{-}} \FunctionTok{sqrt}\NormalTok{(alfa)}\SpecialCharTok{/}\NormalTok{lambda}
\NormalTok{m }\OtherTok{\textless{}{-}} \DecValTok{1000}

\FunctionTok{histogramy}\NormalTok{(alfa,lambda,t0,m,n\_list)}
\end{Highlighting}
\end{Shaded}

\includegraphics{Analiza_przeżycia_5_files/figure-latex/unnamed-chunk-7-1.pdf}
\includegraphics{Analiza_przeżycia_5_files/figure-latex/unnamed-chunk-7-2.pdf}
\includegraphics{Analiza_przeżycia_5_files/figure-latex/unnamed-chunk-7-3.pdf}

\end{document}
